\chapter{Conception de la solution}
\label{Conception de la solution}



Dans cette partie, nous allons commencer d’abord par une vue architecturale de
l’application, puis nous allons détailler la conception en présentant les
diagrammes de séquence et les diagrammes de classe. Ces diagrammes permettent de visualiser les différentes étapes du processus et la structure du système.

\pagebreak
\section{Prototypage}

Le prototypage est une étape importante dans le processus de développement de notre application. Il permet de créer une version préliminaire de
 l'application pour tester ces fonctionnalités, recueillir des retours utilisateurs, et identifier d'éventuelles améliorations avant de passer à la phase de développement complète.

 Dans cette section, nous décrirons les étapes de création du prototype

\subsection{Choix de l'outil}

\begin{figure}[H]
    \centering
    \includegraphics[scale=0.05]{Logos/Figma.png}
    \caption{Logo Figma}
\end{figure} 

Figma est une plateforme collaborative pour éditer des graphiques vectoriels et faire du prototypage. Elle permet de concevoir des design systems pour faciliter la création de sites web et d’applications mobiles. C’est une solution à destination des UI et UX designers et des développeurs. L’interface propose de nombreuses fonctionnalités :

\begin{itemize}
    \item[$\bullet$] \textbf{Design} :avec des outils de conception pour le web, des fonctions de mise en page automatique, des plugins pour réduire les tâches répétitives.
    \item[$\bullet$] \textbf{Prototypage} : pour tester les concepts très tôt en cours de design.
    \item[$\bullet$] \textbf{Design system} : pour concevoir des design cohérents avec des bibliothèques mises à jour en permanence.
    \item[$\bullet$] \textbf{Collaboratives} : pour travailler à plusieurs et en même temps sur un projet, revenir sur une version antérieure si nécessaire ou encore afficher le travail d’un seul collaborateur par exemple.
\end{itemize}

\subsection{Interface}

\subsubsection{Espace apprenant}

\begin{figure}[H]
    \centering
    \begin{minipage}{0.45\textwidth}
        \centering
        \includegraphics[width=\textwidth]{Figures/login.PNG}
        \caption{Figma: Page Login}
    \end{minipage}
    \hfill
    \begin{minipage}{0.45\textwidth}
        \centering
        \includegraphics[width=\textwidth]{Figures/home.PNG}
        \caption{Figma: Page Home}
    \end{minipage}
\end{figure}

\begin{figure}[H]
    \centering
    \begin{minipage}{0.45\textwidth}
        \centering
        \includegraphics[width=\textwidth]{Figures/explorer.PNG}
        \caption{Figma: Page des Formations}

    \end{minipage}
    \hfill
    \begin{minipage}{0.45\textwidth}
        \centering
        \includegraphics[width=\textwidth]{Figures/watchVideo.PNG}
        \caption{Figma: Page regarder une vidéo }
        
    \end{minipage}
\end{figure}


\begin{figure}[H]
    \centering
    \begin{minipage}{0.45\textwidth}
        \centering
        \includegraphics[width=\textwidth]{Figures/overview.PNG}
        \caption{Figma: Page Overview}
    \end{minipage}
    \hfill
    \begin{minipage}{0.45\textwidth}
        \centering
        \includegraphics[width=\textwidth]{Figures/Chapter.PNG}
        \caption{Figma: Page des Chapitres }
    \end{minipage}
    
\end{figure}

\begin{figure}[H]
    \centering
    \begin{minipage}{0.45\textwidth}
        \centering
        \includegraphics[width=\textwidth]{Figures/AuthorDetails.PNG}
        \caption{Figma: Page detail d'un formateur}
    \end{minipage}
    \hfill
    \begin{minipage}{0.45\textwidth}
        \centering
        \includegraphics[width=\textwidth]{Figures/comment.PNG}
        \caption{Figma: Page des ChapitresPage }
    \end{minipage}
    
\end{figure}

\subsubsection{Espace administrateur}

\begin{figure}[H]
    \centering
    \begin{minipage}{0.45\textwidth}
        \centering
        \includegraphics[width=\textwidth]{Figures/figmaDashoboard.PNG}
        \caption{Figma: Page tableau de board}
    \end{minipage}
    \hfill
    \begin{minipage}{0.45\textwidth}
        \centering
        \includegraphics[width=\textwidth]{Figures/addCourseDashboard.png}
        \caption{Figma: Page d'ajout de formation}
    \end{minipage}
    
\end{figure}

\section{Architecture de l’application}

\subsection{Architecture physique}

Nous avons opté pour l’architecture client/serveur multi-tiers. En effet, l’accès à l’application exige le passage à
travers des requêtes HTTP afin de récupérer et de déposer des versions dans le dépôt
central. De plus, la gestion de la base de données du système doit être centralisée et délocalisée de l’endroit de la couche métier, ce qui aide à garder une aisance de maintenance.
Et enfin, il faut que l’application soit distribuée sur plusieurs serveurs et chaque serveur
s’occupe d’une tâche. En effet, grâce au partage des tâches entre les différents serveurs,
nous pourrons garantir une grande souplesse, des bonnes performances et un temps de
réponse réduit. La figure suivante illustre l’architecture physique que nous avons :

\begin{figure}[H]
    \centering
    \includegraphics[width=15cm]{Figures/architecturePhysique.png}
    \caption{Architecture physique de système.}
\end{figure}

Cette architecture se compose principalement des éléments suivants :

\begin{itemize}
    \item[$\bullet$] \textbf{Serveur REST} : Un serveur web qui suit les principes de l'architecture REST et expose des ressources via des URI, permettant aux clients d'effectuer des opérations standardisées sur ces ressources pour accéder aux données et fonctionnalités du serveur.
    \item[$\bullet$] \textbf{Serveur 4D} : Ce serveur contient la couche métier de notre application.
    \item[$\bullet$] \textbf{Serveur de base de données} : Ce serveur se charge de la gestion du stockage des données.
    \item[$\bullet$] \textbf{Couche réseau} : Le protocole TLS sécurise les connexions client/serveur en cryptant les données échangées, permettant ainsi de renforcer la sécurité de votre application 4D Server.
\end{itemize}

\subsection{Architecture logique}

Pour avoir une architecture robuste, modulable et évolutive, il faut utiliser le principe
de « Couche », et donc séparer au maximum les différents types de traitement de l’application. L’environnement de travail n’est pas dépendant à une technologie spécifique.
Pour cette raison, nous avons utilisé plusieurs technologies afin de développer une solution aboutie, performante, multicouches et qui s’intègre parfaitement. La figure suivante
illustre l’architecture logicielle proposée pour le système développé, en présentant trois
couches : couche présentation (web), couche métier qui s’occupe des différents traitements
et couche accès aux données.


\begin{figure}[H]
    \centering
    \includegraphics[width=9cm]{Figures/architectureLogique.png}
    \caption{Architecture logique de système.}
\end{figure}

Au niveau 4D Server, notre développement s’est concentré principalement sur la couche
métier. En effet, 4D Server offre un environnement de développement qui simplifie considérablement la création d’applications. Les autres couches, telles que la couche d’accès
aux données et la couche de présentation, sont déjà implémentées et intégrées dans 4D.Ainsi, les développeurs peuvent se concentrer sur la logique métier de leurs applications
sans avoir à se soucier des détails techniques des autres couches. Cette approche permet
un développement rapide et efficace, tout en offrant des fonctionnalités avancées pour
répondre aux besoins spécifiques des projets.

Aussi, nous avons travaillé avec ORDA, est une technologie spécifique qui facilite
l’accès à une base de données relationnelle en tant qu’objets. Elle permet de manipuler les
données de la base de données à l’aide d’un langage de programmation orienté objet ou
d’interfaces utilisateur spécifiques. ORDA simplifie l’interaction avec la base de données
en fournissant des abstractions supplémentaires et en masquant certaines complexités liées
aux requêtes SQL.

ORDA nous permet de créer des fonctions de classe de haut niveau au-dessus du modèle de données. Cela nous permet d'écrire du code orienté métier et de le «publier» comme une API. Le datastore, les dataclasses, les entity selections et les entités sont tous disponibles en tant qu'objets de classe pouvant contenir des fonctions.

\begin{figure}[H]
    \centering
    \includegraphics[width=15cm]{Figures/orda.png}
    \caption{Orda Data model classe.}
\end{figure}

Grâce à 4D, les développeurs peuvent se concentrer sur l’essentiel et créer des applications puissantes et performantes en toute simplicité.

\subsection{architecture techniques}

\begin{figure}[H]
    \centering
    \includegraphics[width=10cm]{Figures/architectureTechnique.png}
    \caption{Architecture technique de système.}
\end{figure}

Le connecteur s’occupe de la récolte des données saisies par l’utilisateur dans le navigateur, ces données sont envoyées au serveur 4D via des requêtes HTTP. La couche web
récupère les données reçues et les transmet à la couche métier qui effectue les traitements
nécessaires. La couche 4D Database s’occupe de la sérialisation et la dé-sérialisation.
\section{Conception Détaillée}

Le diagramme de classe est l’un des diagrammes statiques d’UML. Il permet de décrire
la structure d’un système informatique tout en montrant les différentes classes, leurs
attributs, leurs méthodes ainsi que les relations entre eux.

\begin{figure}[H]
    \centering
    \includegraphics[width=19cm]{Figures/diagramme de classe.PNG}
    \caption{Diagramme de Classe}
\end{figure}

\subsection{Diagramme de séquence}

Un diagramme de séquence est un diagramme UML (Unified Modeling Language) qui représente la séquence de messages entre les objets au cours d'une interaction.

\subsubsection{Diagramme de séquence de l’authentification}

L’authentification est l’étape primordiale pour toutes les fonctionnalités. L'interaction débute par l’utilisateur qui saisit ses informations d'authentification ( username et mot de passe) et les envoie au système.
Dans le premier cas, le système transmet ces informations à la base de données pour vérification. Si les informations sont correctes, le système redirige l'utilisateur vers la page d'accueil.
Dans le deuxième cas, si les informations d'authentification sont incorrectes, la base de données envoie une réponse négative au système. Le système informe alors l'utilisateur que les informations saisies sont incorrectes, et il est redirigé vers la page d'authentification pour qu'il puisse réessayer.


\begin{figure}[H]
    \centering
    \includegraphics[width=19cm]{Figures/diagrammeDeSequenceAuthentification.png}
    \caption{Diagramme de séquence de l'authentification}
\end{figure}

\subsubsection{Diagramme de séquence d’ajouter une formation}

Ce diagramme montre comment un administrateur ajoute une formation. L'administrateur clique sur "ajouter formation", un formulaire s'affiche, et les informations sont saisies. Le processus se poursuit avec l'ajout de chapitres et de vidéos, l'importation des vidéos et l'envoi des données pour la création de la formation. Une notification est affichée pour indiquer si la création a réussi ou échoué.

\begin{figure}[H]
    \centering
    \includegraphics[width=15cm]{Figures/diagrammeDeSequenceCreerCourse.PNG}
    \caption{Diagramme de séquence d'ajouter une formation}
\end{figure}

\subsubsection{Diagramme de séquence de commencer une formation}

Ce diagramme illustre le processus d'un apprenant qui commence un cours. L'apprenant clique sur "commencer un cours", une requête est envoyée pour récupérer les chapitres et les vidéos du cours, et les vidéos sont ensuite lues.

\begin{figure}[H]
    \centering
    \includegraphics[width=19cm]{Figures/diagrammeDeSequenceCommencerCours.PNG}
    \caption{Diagramme de séquence de commencer une formation}
\end{figure}

\subsubsection{Diagramme de séquence d'ajouter un formateur}

Le diagramme de séquence montre le processus par lequel un administrateur ajoute un formateur. L'administrateur clique sur "ajouter formateur" et le système affiche un formulaire à remplir. Après la soumission du formulaire, le système envoie une requête à la base de données pour créer le formateur. Si la création réussit, le système notifie l'administrateur que le formateur a été ajouté avec succès. En cas d'échec, une notification d'échec est envoyée.

\begin{figure}[H]
    \centering
    \includegraphics[width=19cm]{Figures/diagrammeDeSequenceAddAuthor.PNG}
    \caption{Diagramme de séquence d'ajouter formateur}
\end{figure}

\newpage
\subsection*{Conclusion}

Au cours de ce chapitre, nous avons présenté tout d'abord le prototype de notre application, après nous avons vu les architectures utilisées et des diagrammes
de classes et de séquences pour mieux comprendre notre projet. Nous pouvons alors passer à l’implémentations et à la validation
de notre solution.

