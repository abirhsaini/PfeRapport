\chapter*{Remerciements}
\addcontentsline{toc}{chapter}{Remerciements}


Tout d’abord, je remercie Allah le tout-puissant de m’avoir donné le courage et la patience nécessaires à mener ce travail à son terme.

\vspace{10pt}
Je tiens à remercier toutes les personnes qui par leur support ou par leur simple présence, m'ont permis de rendre mon travail aussi instructif et bénéfique que plaisant.

\vspace{10pt}
Je transmets mes sincères remerciements à mon encadrant \textbf{M. HAFIDDI \textsc{Hatim}} qui m'a allégé par ses connaissances, son savoir et ses précieuses orientations, pour les conseils qu’il a prodigués, et le suivi pertinent, mais aussi pour la fierté et l’ambition que j'ai acquises suite à l’encouragement et le support intensif, ainsi que l’assistance précieuse.

\vspace{10pt}
Je tiens également à remercier, chaleureusement, mon encadrant externe, \textbf{M. METWALLI \textsc{Ayoub}} pour sa confiance, sa collaboration et son soutien, et de m’avoir supervisé tout au long de mon stage par ses conseils, ses orientations et son sens de rigueur qui m’a guidé durant ce travail et de m’avoir permis de m’intégrer. Je tiens aussi à témoigner toute ma reconnaissance aux membres de l’entreprise 4D, pour l’expérience enrichissante et pleine d’intérêt qu’ils m’ont fait vivre durant ces mois de stage parmi eux.

\vspace{10pt}
Je tiens à remercier également les membres de jury, \textbf{M. LAGHOUAOUTA \textsc{Youness}} et Monsieur \textbf{M. CHAMI \textsc{Mouhcine}}, pour l’évaluation du travail réalisé. Mes remerciements s’adressent aussi à l’ensemble du corps enseignant de l’institut national des postes et télécommunications (INPT), pour le temps qu’ils ont consacré pour nous offrir une formation d’excellence et de polyvalence et à toutes les personnes qui nous ont été d’une aide précieuse.
