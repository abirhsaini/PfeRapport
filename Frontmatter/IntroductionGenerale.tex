\chapter*{Introduction générale}
\addcontentsline{toc}{chapter}{Introdcution}


Dans un monde avec une grande évolution, où les technologies changent rapidement, la nécessité d'une formation continue efficace, simple et accessible est devenue primordiale pour rester présent sur le marché du travail. Les entreprises et les individus reconnaissent de plus en plus l’importance d'une formation continue en certaines domaines pour améliorer leurs compétences, acquérir de nouvelles connaissances et s'adapter aux changements rapides de l'environnement professionnel.\\

C'est dans ce contexte que s'inscrit notre projet de fin d'études (PFE), qui vise à créer une plateforme de formation continue adaptée aux besoins actuels des apprenants et les clients de 4D. L’objectif principal de notre projet est de fournir une solution technologique robuste et conviviale qui facilite l'accès aux différentes formations du langage 4D ainsi que les autres langages, tout en offrant des fonctionnalités avancées pour suivre, évaluer et personnaliser le processus d'apprentissage.\\

Ce rapport présente une vue d'ensemble détaillée du processus de développement de notre plateforme de formation continue, composé de quatre chapitres. Dans le premier chapitre, nous définissons le contexte général du projet, à savoir, la présentation de l’organisme d’accueil 4D logiciel, le contexte générale et la conduite du projet. Le deuxième chapitre est consacré à l’analyse et la spécification des besoins pour développer cette plateforme. Le troisième chapitre est consacré à la définition des architectures utilisées, ainsi que la modélisation des diagrammes de classes et de séquence. Finalement, dans le quatrième chapitre, nous parlons de l’implémentation et la validation de la solution suivie par la conclusion générale où nous abordons des perspectives d’évolution.\\

En fin de compte, notre objectif est de créer une plateforme qui favorise l'échange de connaissances et le développement professionnel, contribuant ainsi à renforcer les compétences des employés de 4D et  et à soutenir l'innovation au sein des organisations.